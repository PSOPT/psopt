\hypertarget{index_intro}{}\section{Introduction}\label{index_intro}
The author developed the main features of the \hyperlink{classDMatrix}{DMatrix} class between 1994 and 1999. The class makes extensive use of operator overloading in order to facilitate the implementation in C++ of complicated matrix expressions, and it has intefaces to a number of LAPACK routines. In 2008, the class has been tested with current compilers, its functionality was expanded, and the code was published under the GNU Lesser General Public License. The class at present is restricted to dense and real matrices. In 2008, the SparseMatrix class was added to the library to incorporate basic sparse matrix functionality. The SparseMatrix class offers interfaces to some functions available in the CXSparse and LUSOL libraries.

The library should compile without problems with the following C++ compilers: GNU C++ version 4.X and Microsoft Visual Studio 2005.\hypertarget{index_license}{}\section{License}\label{index_license}
This work is copyright (c) Victor M. Becerra (2009)

This library is free software; you can redistribute it and/or modify it under the terms of the GNU Lesser General Public License as published by the Free Software Foundation; either version 2.1 of the License, or (at your option) any later version.

This library is distributed in the hope that it will be useful, but WITHOUT ANY WARRANTY; without even the implied warranty of MERCHANTABILITY or FITNESS FOR A PARTICULAR PURPOSE. See the GNU Lesser General Public License for more details. You should have received a copy of the GNU Lesser General Public License along with this library; if not, write to the Free Software Foundation, Inc., 51 Franklin Street, Fifth Floor, Boston, MA 02110-\/1301 USA, or visit \href{http://www.gnu.org/licenses/}{\tt http://www.gnu.org/licenses/}

Author: Dr. Victor M. Becerra, University of Reading, School of Systems Engineering, P.O. Box 225, Reading RG6 6AY, United Kingdom, e-\/mail: \href{mailto:v.m.becerra@ieee.org}{\tt v.m.becerra@ieee.org}.\hypertarget{index_install}{}\section{Installing the library}\label{index_install}
See the INSTALL file. \hypertarget{index_examples}{}\section{Examples of use}\label{index_examples}
See the source code in the examples directory. 